\documentclass{ctexart}
\usepackage{listings}
\usepackage{xcolor}
\usepackage{fontspec}
\usepackage{hyperref}

% 设置代码字体为 Code New Roman Nerd Font
\setmonofont{CodeNewRoman Nerd Font}

% 超链接样式配置
\hypersetup{
    colorlinks=true,
    linkcolor=blue,
    urlcolor=blue,
    citecolor=blue
}

% 定义 Rust 语言
\lstdefinelanguage{Rust}{
  keywords={as, break, const, continue, crate, else, enum, extern, false, fn, for, if, impl, in, let, loop, match, mod, move, mut, pub, ref, return, self, Self, static, struct, super, trait, true, type, unsafe, use, where, while, async, await, dyn},
  keywordstyle=\color{blue}\bfseries,
  ndkeywords={bool, u8, u16, u32, u64, u128, i8, i16, i32, i64, i128, f32, f64, usize, isize, str, char, Box, Vec, String, Option, Result},
  ndkeywordstyle=\color{purple}\bfseries,
  sensitive=true,
  comment=[l]{//},
  morecomment=[s]{/*}{*/},
  commentstyle=\color{gray}\itshape,
  stringstyle=\color{red},
  morestring=[b]",
  morestring=[b]'
}

% 代码块样式
\lstset{
  basicstyle=\ttfamily\small,
  keywordstyle=\color{blue}\bfseries,        % 关键字:蓝色粗体
  commentstyle=\color{gray}\itshape,         % 注释:灰色斜体
  stringstyle=\color{red},                   % 字符串:红色
  numberstyle=\tiny\color{gray},             % 行号:灰色
  identifierstyle=\color{black},             % 标识符:黑色
  backgroundcolor=\color{gray!10},           % 背景色
  frame=single,
  xleftmargin=2em,
  framexleftmargin=1.5em,
  breaklines=true,
  showstringspaces=false                     % 不显示字符串中的空格
}

\begin{document}

\title{ch1-应用程序与基本执行环境}
\author{odlove}
\date{\today}
\maketitle

\section{Introduction}

本章的最终任务是完成一个基于
\href{https://github.com/rustsbi/rustsbi}{RustSBI}
服务和
\href{https://www.qemu.org/}{QEMU}
完成输出 Hello, world 并关机.

\section{Content}

\subsection{第一步: 创建一个裸机程序}

创建工作目录
\begin{lstlisting}[language=bash]
$ mkdir rCore
$ cd rCore
$ cargo new os
$ cd os
\end{lstlisting}

此时我们的工作目录已经创建完成,
运行
\begin{lstlisting}
$ cargo run
\end{lstlisting}
可以看到输出的 ``Hello, world!".

但是我们的目标是编译到 riscv 的裸机环境,
cargo 使用 \verb|--target| 来指定编译目标,
所以让我们运行
\begin{lstlisting}[language=bash]
$ cargo build --target riscv64gc-unknown-none-elf
\end{lstlisting}

运行后会看到如下错误输出:
\begin{lstlisting}
error[E0463]: can't find crate for `std`
  |
  = note: the `riscv64gc-unknown-none-elf` target may not support the standard library
  = note: `std` is required by `os` because it does not declare `#![no_std]`

error: cannot find macro `println` in this scope
 --> src/main.rs:2:5
  |
2 |     println!("Hello, world!");
  |     ^^^^^^^

error: `#[panic_handler]` function required, but not found

For more information about this error, try `rustc --explain E0463`.
error: could not compile `os` (bin "os") due to 3 previous errors
\end{lstlisting}

这里有三个错误:
\begin{enumerate}
    \item{找不到标准库}
        \begin{lstlisting}
error[E0463]: can't find crate for `std`
  |
  = note: the `riscv64gc-unknown-none-elf` target may not support the standard library
  = note: `std` is required by `os` because it does not declare `#![no_std]`
        \end{lstlisting}
        这是因为我们使用的是裸机环境,
        std 库是操作系统之上的一层抽象,
        依赖于操作系统.
    \item{找不到 \verb|println| 宏}
        \begin{lstlisting}
error: cannot find macro `println` in this scope
 --> src/main.rs:2:5
  |
2 |     println!("Hello, world!");
  |     ^^^^^^^
        \end{lstlisting}
        println 宏来自于 std 库,
        没有 std 库自然就没有 println 宏
    \item 缺少 \verb|#[panic_handler]|
        \begin{lstlisting}
error: `#[panic_handler]` function required, but not found
        \end{lstlisting}
        \verb|#[panic_handler]| 负责在rust程序崩溃时进行处理,
        一般情况下 std 库会提供, 但是我们没有 std 库.
        这是 rust 语言本身的要求, 参见
        \href{https://doc.rust-lang.org/reference/panic.html\#the-panic\_handler-attribute}{Rust Reference}
        和
        \href{https://docs.rust-embedded.org/book/start/panicking.html}{Embedded Rust Book}.
\end{enumerate}

在解决这些问题之前, 让我们使用一个小小的 trick:
在当前目录下创建 \verb|.cargo/config.toml| 文件, 内容如下:
\begin{lstlisting}
[build]
target = "riscv64gc-unknown-none-elf"
\end{lstlisting}
这将会更改 \verb|cargo| 的默认目标,
我们就不需要每次在命令后添加
\texttt{-{}-target riscv64gc-unknown-none-elf} 了.

\verb|println| 宏的问题最好解决, 只要直接删除就好了.
标准库的问题则是要在程序开头添加 \verb|#![no_std]|.
而\verb|#[panic_handler]|则需要我们添加一个函数:
\begin{lstlisting}[language=Rust]
use core::panic::PanicInfo;

#[panic_handler]
fn panic(_info: &PanicInfo) -> ! {
    loop {}
}
\end{lstlisting}

再次编译, 又遇到了新的错误:
\begin{lstlisting}
error: using `fn main` requires the standard library
  |
  = help: use `#![no_main]` to bypass the Rust generated entrypoint and declare a platform specific entrypoint yourself, usually with `#[no_mangle]`

error: could not compile `os` (bin "os") due to 1 previous error
\end{lstlisting}
这是因为 \verb|main| 函数的环境也是由 std 库提供的,
关于这一点我们会在后面[程序是如何在计算机上运行的]说明.
现在我们需要添加 \verb|#![no_main]| 并删除 \verb|main| 函数.

此时的 \verb|main.rs| 内容如下:
\begin{lstlisting}[language=rust]
#![no_std]
#![no_main]

use core::panic::PanicInfo;

#[panic_handler]
fn panic(_info: &PanicInfo) -> ! {
    loop {}
}
\end{lstlisting}

此时运行 \verb|cargo build|, 我们终于编译成功了.

让我们来看看此时编译得到的可执行文件:
\begin{lstlisting}[language=bash]
$ file target/riscv64gc-unknown-none-elf/debug/os
\end{lstlisting}
得到:
\begin{lstlisting}
target/riscv64gc-unknown-none-elf/debug/os: ELF 64-bit LSB executable, UCB RISC-V, RVC, double-float ABI, version 1 (SYSV), statically linked, with debug_info, not stripped
\end{lstlisting}
\verb|ELF 64-bit|: 64位ELF格式\\
\verb|LSB|: Little-Endian(小端序)\\
\verb|UCB RISC-V|: RISC-V架构\\
\verb|RVC|: 支持压缩指令集(C扩展)\\
\verb|double-float ABI|: 双精度浮点\\
\verb|statically linked|: 静态链接\\
\verb|with debug_info|: 包含调试信息\\
\verb|not stripped|: 符号表还在

运行
\begin{lstlisting}
$ rust-objdump -h target/riscv64gc-unknown-none-elf/debug/os
\end{lstlisting}
得到
\begin{lstlisting}
target/riscv64gc-unknown-none-elf/debug/os:     file format elf64-littleriscv

Sections:
Idx Name              Size     VMA              Type
  0                   00000000 0000000000000000
  1 .debug_abbrev     00000106 0000000000000000 DEBUG
  2 .debug_info       000005a9 0000000000000000 DEBUG
  3 .debug_aranges    00000030 0000000000000000 DEBUG
  4 .debug_str        0000047b 0000000000000000 DEBUG
  5 .comment          0000008b 0000000000000000
  6 .riscv.attributes 00000066 0000000000000000
  7 .debug_frame      00000038 0000000000000000 DEBUG
  8 .debug_line       00000048 0000000000000000 DEBUG
  9 .symtab           00000138 0000000000000000
 10 .shstrtab         00000083 0000000000000000
 11 .strtab           00000051 0000000000000000
\end{lstlisting}
\verb|objdump -h| 是查看 Section Headers 的命令.

此处需要注意的是这个 elf 文件没有
\verb|.text| \verb|.rodata| \verb|.data| \verb|.bss| 段,
这说明目前我们的程序是一个空程序.
并且所有的 VMA 都为 0,
这表明链接器不知道这个程序应该被加载到内存的什么地址.
在裸机环境中, 我们需要通过链接脚本(linker script)来告诉链接器
代码和数据应该放在内存的哪个位置.


\end{document}
